\chapter{Krav}
\section{Krav til ekstern grænseflade}
\subsection{GUI}
Softwaren giver en god grafisk grænseflade for brugeren, og administratoren kan operere på systemet og udføre den krævede opgave, f.eks. Oprette, opdatere, se detaljerne i forhold til et projekt som er blevet tilføjet.

\begin{itemize}  
    \item Det giver lagerbekræftelse og søgefacilitet baseret på forskellige kriterier.
    \item Designet skal være enkelt, og alle de forskellige grænseflader skal følge en standardskabelon.
    \item Alle moduler, der følger med softwaren, skal passe ind i denne grafiske brugergrænseflade og udføre til den definerede standard.
    \item Brugergrænsefladen skal være i stand til at interagere med brugeradministrationsmodulet, og en del af grænsefladen skal være dedikeret til de forskellige moduler som sytemet består af.
\end{itemize}

\subsection{Home page}
Den første side brugeren tilgå, når man gå ind på hjemmesiden.
\subsection{Login}
Hvis Brugerens konto er oprettet, kan han/hun 'Login', som beder brugeren om at indtaste sit brugernavn og adgangskode.

\subsection{Signup}
 Hvis brugeren ikke er registreret endnu, kan han indtaste detaljerne og registrere for at oprette sin konto.

 \subsection{Forgot password}
 Hvis brugeren indtastede enten sit brugernavn eller adgangskode forkert, er der en fejlmeddelelse kommer til syne. Dermed kan brugeren nulstille sin adgangskode via forgot password siden. 

 \subsection{Categories View}
Visning af kategorier viser de tilgængelige kategorier af jobs og giver brugeren mulighed for at finde projekter indenfor det han/hun er interesseret i.

\subsection{About}
 About siden fortæller de ny ankommet bruger om hvordan platformen fungere og hvordan det er at være freelancer og medarbejder hos converge.

 \subsection{Search}
Brugeren har mulighed for at søge efter det kategori han/hun er interesseret i ved at indtaste katogri navn.

\subsection{Video chat}
Brugeren kan komunikere over video chat, hvor der kan ske udvikling af Information.

\subsection{Chat}
Brugeren kan chatte med andre brugere.

\subsection{Settings}
Her vil der mulighed for at man som brugere kan fortage personlige ændringer.
\subsection{Payments}
Siden viser hvor mange penge brugeren har på sin konto.

\subsection{Jobs}
Siden viser en liste med de forskellige jobs der er på nuværende tidspunkt og dem kan brugeren så byde på.
\subsection{Post Project}
På denne side kan der oprettes projekt som man ønsker få udviklet af en brugere fra platformen.
\subsection{Portfolio}
Viser information om brugerens kompentancer.
\subsection{Project page}
Her Vil der være muglighed for at kunne se en tidslinje over projektets fremgang.
