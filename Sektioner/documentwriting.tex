\subsection{Dokumentskrivning}

Til dokumentskrivning var der en del alternativer såsom

\begin{itemize}
    \item MSWord
    \item Google Docs
    \item \LaTeX
    \item Markdown + Pandoc
    \item Asciidoc
    \item GitBooks v1
\end{itemize}

\subsubsection{MSWord}

Word er en Wysiwyg der er nem at skrive i, men egner sig ikke specielt godt til mange versionskontrol systemer, da det gemmes som en binær fil, og er ikke specielt anvendeligt i forhold til reference lister og figur tekster. Det er også meget svært at få finpudset sit dokument, da Word kan være svært at manipulere.

\subsubsection{Google Docs}

Word er en kopi af Word, og har samme styrker og svagheder, men har tilgængeld indbygget versionering og samarbejde egenskaber. Men de samme svagheder som Word har, gør det til en dealbreaker.

\subsubsection{\LaTeX}

LaTeX er et fantastisk værtøj til at skrive dokumenter med, men er mest beregnet til akademiske værker. Og har en meget højere indlærings kurve end de andre på listen. LaTeX bruger et avanceret plugin system til at producere dokumentation, så at sætte det op er ikke trivielt. Men vil uden tvivl give det bedste slutresultat i forhold til ønske og effekt.

\subsubsection{Markdown + Pandoc}

Markdown er et sprog der er lavet til at være letlæseligt, men er ikke specielt godt egnet til større dokumentationer. Det er især nyttigt, hvis man er god til at lave sin dokumentation til sektioner der ikke er ret lange (mere end et par sider). Pandoc bruges til at kompilere markdown til html og derved PDFs eller LaTeX for bedre manipulation af resultat. Markdown egner sig ikke generelt til dokument skrivning, men med andre værktøjer kan det lade sig gøre, specielt ved brug af LaTeX sammen med.

\subsubsection{Asciidoc}

Asciidoc er en anden måde at skrive dokumentation på, det ligner meget markdown i dets målgruppe, men har mere avancerede værktøjer og egenskaber, ikke lige så mange som LaTeX, men flere end markdown. Sproget er ikke understøttet ret mange steder mere, og enten skriver man i Markdown eller LaTeX. 

\subsubsection{GitBooks v1}

GitBooks v1, var en seludgivelses platform, som integerede med git, til at udgive bøger skrevet i markdown, LaTeX eller Asciidoc. Denne platform, ville være til at foretrække, hvis ikke at firmaet bag, har ændret deres mission, i stedet for selvudgivelse har de skiftet over til projekt dokumentation ligesom Confluence (Atlassian).

\subsubsection{Valg}

Vi har valgt at skrive i LaTeX, vi havde allerede nogle templates (opsætning) vi kunne bruge fra tidligere semestre, så opstarten, var ikke nær så høj, som hvis det var første gang sproget blev brugt. Desuden bruges der git til versionskontrol, og LaTeX virker super med git, eftersom der ikke er nogle binære filer, i modsætning til Word, Docs. Sproget er også avanceret, og gør mange svære opgaver trivielle, såsom referencer, lister, tabeller osv.

