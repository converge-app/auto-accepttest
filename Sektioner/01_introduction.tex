\section{Indledning}

I dette dokument vil autoaccepttests blive beskrevet for Converge produktet.

\section{Formål}

Auto accepttest har som formål at automatiserer processen ved at verificerer kunde krav bekrevet som user stories. Derfor benytter Converge automatiserede værktøjer til at tjekke disse krav. Auto accepttests fungerer som systemtests, men tjekker kun de faktiske krav, og er ikke en erstatning for systemtests.

\section{Opdeling}

Autoaccepttests er opdelt efter epics, hvilket kan findes i kravspecifikationen \cite{documentation-kravspec}. De aktuelle tests er beskrevet som user stories og de enkelte tests er opbevaret i Converge-SPA repositoriet \cite{repository-converge-spa}.

\section{Coded UI tests}

De forskellige tests fungererer ved at anvende et framework kaldet Selenium, med Selenium kan man lave kodet UI tests til Websites. Dette har været brugt til at tests funktionalitet.

\subsection{Fremgangsmåde}

Fremgangsmåden er at optage en handling, modificerer den og gøre den reproducerbar. 