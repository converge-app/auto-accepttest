Til client udvikling er der en del alternativer såsom

\begin{itemize}
    \item React
    \item Angular
    \item Vue
    \item Blazor
    \item Asp.net core
    \item Vanilla html
    \item Android \& IOS
\end{itemize}

\subsubsection{React}

React er en SPA (Single page application) og gør at hjemmesider som normalt er statiske, kan opføre sig som en desktop applikation. React er det mest bruge SPA, og der er stor udvikling og værktøjer inden for dette felt.

Desuden har React og en masse plugins, såsom NextJS, til at fungere som applikations server. React har en meget lav indlæringskurve og er trivielt at starte op med. Dog kan mange af de udvidelser brug være meget komplekse, såsom NextJs, Formik og Redux. 

Alt i alt er React et simpelt bibliotek der kan det Converge skal kunne, udvikle en dynamisk platform til projektsamarbejde.

\subsubsection{Angular}

Angular ligner meget React, men har mere komplet tooling uden en masse plugins. Angular er derfor også en del sværre at starte op med. Men det er stadig et rigtigt populært framework.

\subsubsection{Vue}

Vue ligner igen meget React og Angular, men prøver at få det bedste fra begge frameworks. Det er ikke helst så populært, men der er et sundt fællesskab til det.

\subsubsection{Blazor}

Blazor er Microsofts svar på et server drevet SPA. Det er stadig i den eksperimentale phase af dets udvikling, men viser et stort potentiale. Det ville være ideelt hvis det var et mere modent produkt.

\subsubsection{Asp.net Core}

Med Razor engine er asp.net core et godt framework til at lave en server drevet platform. Det er der mange der gør. Desværre er razor pages ikke specielt dynamiske, og kræver mange hacks i koden før at det bliver godt at bruge til converges formål. Det er dog et godt produkt, som ville egne sig godt til f.eks. admin interfaces eller ligende.

\subsubsection{Vanilla html}

Med Vanilla HTML menes der ren html serveret af en reverse proxy som nginx eller apache. Det ville være sværre at bruge end ASP.NET core, så derfor er det ikke relevant.

\subsubsection{Android \& IOS}

Teamet har erfaring med Android og er det eneste udover React som teamet har kendskab til. Android og IOS er primært lavet til mobile enheder der køre Android eller IOS. Derfor er det ikke relevant for projektets formål, eftersom den primære målgruppe vil være desktop brugere.

\subsubsection{Valg}

React er valgt, da det bibliotek, har alt hvad teamet skal bruge i en fornuftig pakke, det gør at teamet kan vælge de værktøjer de gerne vil bruge, samtidig med at have en fornuftig indlæringskurve, så teamet kan overskue at bruge produktet.